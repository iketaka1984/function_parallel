\documentclass{jarticle}
\setlength{\textwidth}{17cm}
\setlength{\textheight}{23.5cm}
\hoffset=-25mm \voffset=-25mm

\usepackage{amsmath}
\usepackage{amsfonts}
\usepackage[cmtip,all]{xy}
\usepackage{url}
\newcommand{\longsquiggly}{\xymatrix{{}\ar@{~>}[r]&{}}}
\newtheorem{thm}{定理}
\newtheorem{defn}[thm]{定義}
\newcommand{\bcode}[1]{$\mathsf{#1}$}
\newcommand{\brightarrow}[1]{\stackrel{#1}{\longsquiggly}}

\newcommand{\blabel}[1]{\mathrm{b}#1}
\newcommand{\plabel}[1]{\mathrm{p}#1}
\newcommand{\clabel}[1]{\mathrm{c}#1}
\newcommand{\flabel}[1]{\mathrm{f}#1}
\newcommand{\alabel}[1]{\mathrm{a}#1}


\begin{document}
\begin{center}
{\bf\Large PRO研究会{2021-1-(7)}の修正について}\\[10pt]
{\large 池田崇史,結縁祥治(名古屋大学)}
\end{center}
発表資料について5月11日提出の原稿から以下の点について,修正を行いました.

内容的に変更した点は,$\mathbf{while}$ループにおける$wn$という構文要素を削除した
ことから生じる修正です.その他,以下の修正の他,フォントの使い方(an,bn,cn,pn,fnなど)の
統一などを行いました.

\begin{enumerate}
\item \underline{p.1 日本語アブストラクト:4行目}\\
「順方向の抽象命令を逆方向の抽象命令に一対一に」\\
→「順方向の抽象命令を逆順とし,ジャンプ命令と変数更新命令を対応する逆方向の命令に」
\item \underline{p.1 英文アブストラクト:5行目}\\
「to reverse abstract instructions one-to-one」\\
→「in the reversed order with jump and update instructions altered to the corresponding reversing instructions」
\item \underline{p.2 左16行目} 「Janusにおいては」の前に「可逆プログラミング言語」を挿入
\item \underline{p.2 左24行目} 「このため」を削除し,「逆方向の実行のためには」を挿入
\item \underline{p.2 図1(6行目)} $S$の定義中「$\mathbf{while}\ wn\ C\ \mathbf{do}\ P\ \mathbf{od}$」から「$\mathbf{while}\ C\ \mathbf{do}\ P\ \mathbf{od}$」に変更($wn$の削除)
\item \underline{p.2 右最終行}「$bn,an,wn,pn,fn,cn$」を 「$bn,an,pn,fn,cn$ 」に変更($wn$を削除)
\item \underline{p.3 左 2.2節(節内15行目)} 「これによって...特定することができる」を以下に修正\\
「このため,seats>0の条件判定が10行目ないし19行目のseatsの減算に有効になっていないことが特定できる」
\item \underline{p.3 図2内 8行目} $\mathbf{while}\ w1\ (agent1==1)\ \mathbf{do}$ から
         $\mathbf{while}\ (agent1==1)\ \mathbf{do}$ に変更($w1$を削除)\\
\item \underline{p.3 図2内 17行目} $\mathbf{while}\ w2\ (agent2==1)\ \mathbf{do}$ から $\mathbf{while}\ (agent2==1)\ \mathbf{do}$
         に変更($w2$を削除)
\item \underline{p.4 表1内} $\mathsf{w\_label}$ $\mathsf{w\_end}$ 削除し,$\mathsf{nop}$の番号を19に変更
\item \underline{p.4 右 3.1.3節(節内1行目)} 「バイトコードにおいて」を削除し,「並列ブロックから図4(a)のようなバイトコードを生成する.」を挿入
\item \underline{p.4 右 3.1.3節(最初の段落の最後)} 「(図4)」→「(図4(b))」
\item \underline{p.4 右 3.1.3節(3番めの段落の最後)} 「$T(an).last=E_\ell$である.」のあとに,「逆方向に
並列ブロックを実行する場合は,...$T(an)^{-1}_N(i)=(N+1-E_i,N+1-B_i)$である.」を挿入.
\item \underline{p.5 図6内} $\mathsf{w\_label}$, $\mathsf{w\_end}$ に対する $inv$ 定義の削除
\item \underline{p.5 図7の修正}
\begin{list}{}{}
\item ラベルスタックに積まれる値を$(N-a+1,p)$から$(a,p)$に変更
\item 逆方向ジャンプの\texttt{label 0}を\texttt{nop 0}に変更
\item 逆方向ジャンプのrjmp 0をrjmp Nに変更
\end{list}
\item \underline{p.6 図8} \texttt{w\_laxbel}, \texttt{w\_end} $\rightarrow$ \texttt{label}
\item \underline{p.6 左 3.2.1の最後の段落} 図8の説明を差し替え「図8にwhile構造の対応を示す....値スタックに保存される.」
\item \underline{p.7 右 $\mathsf{par:}$} 振舞定義およびその定義を差し替え.(6項組間の関係を8項組間の関係に修正)
\item \underline{p.8 左 $\mathsf{fork:}$} 「$\mathsf{fork}\ an$は,...」以下の説明を差し替え
\item \underline{p.8 左から右} $\mathsf{w\_label:}$, $\mathsf{w\_end:}$の定義を削除
\item \underline{p.8 右 $\mathsf{par:}$} 振舞定義およびその定義を差し替え.(6項組間の関係を8項組間の関係に修正)
\item \underline{p.8 右 $\mathsf{r\_alloc}$} $\xrightarrow{(\mathsf{r\_alloc},x)}_p$を
$\brightarrow{(\mathsf{r\_alloc},x)}_p$に修正
\item \underline{p.9 左 $\mathsf{r\_fork:}$} 「$\mathsf{r\_fork}\ an$は,...」以下の説明を差し替え\\
($T(\alabel{n})^{-1}_N$の定義は3.1.3に移動)
\item \underline{p.9 右 図11内} PC=10とPC=38 \texttt{w\_label wn} から\texttt{label 80} に変更\\
          PC=33とPC=61  \texttt{w\_end wn} から  \texttt{label 80} に変更
\item \underline{p.11 右 図16内} ラベルスタックの左の値を修正\\
(\texttt{bug\_fact}では$N=75$であり,誤って76から引いた数字が示されていたが,p.7の\bcode{label}の振舞定義に
したがってこのように修正した.)
\item \underline{p.11 左 9行目} 「一行目の(72 0.b1.E) はプロセス0 がPC=72」から「一行目の(4 0) はプロセス0 がPC=4」に変更
\item \underline{p.13 左 4番目の段落の前(19行目)} 「ブロック構造を逆方向に時刻する場合には...
アノテーションは不要になっている.」のパラグラフを追加
\item \underline{p.13 左 4番目の段落} 「抽象機械の概念を利用した...」から始まる段落中\\
 p.13 左 20行目「可逆実行環境として,」を削除し,「可逆実行環境が」を挿入
\end{enumerate}

以上,よろしくお願いいたします.
\end{document}
